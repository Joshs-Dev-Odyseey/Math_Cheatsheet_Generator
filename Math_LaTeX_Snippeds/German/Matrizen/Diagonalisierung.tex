\subsubsection{Diagonalisierung einer Matrix}
\textbf{Formel}
\begin{flalign}
    &S^{-1}AS = D = \begin{pmatrix}
        \lambda_1 & \cdots & 0\\
        \vdots & \ddots & \vdots\\
        0 & \cdots & \lambda_n
    \end{pmatrix} \Leftrightarrow A = SDS^{-1}&\label{eq:Defintion_Diagonalmatrix}\\
    &A\underbrace{Se_i}_{\vec{x}_i} = S(De_i) = S \lambda_i e_i = \lambda_i \cdot \underbrace{Se_i}_{\vec{x}_i}&\notag\\
    &\Rightarrow A \vec{x}_i = \lambda_i \vec{x}_i, \vec{x}_i\notag&(\ref{eq:Eigenwertproblem})
\end{flalign}\\

\textbf{Wichtiges}\\
\begin{multicols}{2}
    \begin{itemize}
        \item A: Eine Matrix
        \item S: Matrix mit Eigenvektoren
        \item D: Diagonalmatrix mit Eigenwerten
        \item $e_i$: Spalteneinheitsvektor
    \end{itemize}
\end{multicols}

\textbf{Anleitung}\\
\begin{enumerate}
    \item Eigenwerte ausrechnen \ref{eq:Eigenwertproblem}
    \item Wenn n > 2: Eigenvektoren normieren
    \item Matrizen D und S aufstellen
    \item Inverse von S ausrechnen\\
    \bullet Bei orthogonaler Matrix: $S^{-1} = S^T$\\
    \bullet Wenn n $\ge$ 3: Mit dem Adjunkten Verfahren die Matrix berechnen (\ref{eq:4x4_Inverse_berechnen})\\
    \bullet Wenn n > 3 und Matrix nicht orthogonal: Gram-Schmidt-Orthogonalisierungsverfahren
    \item \ref{eq:Defintion_Diagonalmatrix} aufstellen
\end{enumerate}

\begin{flalign}
    &\boxed{3}&\notag\\
    &\frac{1}{a} \scalebox{0.65}{$\displaystyle \begin{pmatrix}
            \vec{x}_{EV_{1_1}} & \vec{x}_{EV_{2_1}} & \cdots & \vec{x}_{EV_{n_1}}\\
            \vec{x}_{EV_{1_2}} & \vec{x}_{EV_{2_2}} & \cdots & \vec{x}_{EV_{n_2}}\\
            \vdots & \vdots & \ddots & \vdots\\
            \vec{x}_{EV_{1_n}} & \vec{x}_{EV_{2_n}} & \cdots & \vec{x}_{EV_{n_n}}\\
        \end{pmatrix}
        $} \cdot \scalebox{0.65}{$\displaystyle \begin{pmatrix}
            \lambda_{EW_1} & 0 & \cdots & 0\\
            0 & \lambda_{EW_2} & \cdots & 0\\
            \vdots & \vdots & \ddots & \vdots\\
            0 & 0 & \cdots & \lambda_{EW_n}\\
        \end{pmatrix}
        $} \cdot \scalebox{0.65}{$\displaystyle \begin{pmatrix}
            \vec{x}_{EV_{1_1}} & \vec{x}_{EV_{2_1}} & \cdots & \vec{x}_{EV_{n_1}}\\
            \vec{x}_{EV_{1_2}} & \vec{x}_{EV_{2_2}} & \cdots & \vec{x}_{EV_{n_2}}\\
            \vdots & \vdots & \ddots & \vdots\\
            \vec{x}_{EV_{1_n}} & \vec{x}_{EV_{2_n}} & \cdots & \vec{x}_{EV_{n_n}}\\
        \end{pmatrix}^{-1}
        $}&
\end{flalign}

\subsubsection{Matrix Potenzieren}
\textbf{Benötigt}\\
\begin{itemize}
    \item Bei hohen Potenzen wird die Diagonalisierte Matrix benötigt \ref{eq:Defintion_Diagonalmatrix}
\end{itemize}
\textbf{Formel}\\
\begin{flalign}
    &A^m = A \cdot A \cdot \ldots \cdot A = SDS^{-1} \cdot SDS^{-1} \cdot \ldots \cdot SDS^{-1}&\\
    &A^m = SD^mS^{-1}&\notag\\
    &\frac{1}{a} \scalebox{0.65}{$\displaystyle \begin{pmatrix}
            \vec{x}_{EV_{1_1}} & \vec{x}_{EV_{2_1}} & \cdots & \vec{x}_{EV_{n_1}}\\
            \vec{x}_{EV_{1_2}} & \vec{x}_{EV_{2_2}} & \cdots & \vec{x}_{EV_{n_2}}\\
            \vdots & \vdots & \ddots & \vdots\\
            \vec{x}_{EV_{1_n}} & \vec{x}_{EV_{2_n}} & \cdots & \vec{x}_{EV_{n_n}}\\
        \end{pmatrix}
        $} \cdot \scalebox{0.65}{$\displaystyle \begin{pmatrix}
            \lambda_{EW_1}^m & 0 & \cdots & 0\\
            0 & \lambda_{EW_2}^m & \cdots & 0\\
            \vdots & \vdots & \ddots & \vdots\\
            0 & 0 & \cdots & \lambda_{EW_n}^m\\
        \end{pmatrix}
        $} \cdot \scalebox{0.65}{$\displaystyle \begin{pmatrix}
            \vec{x}_{EV_{1_1}} & \vec{x}_{EV_{2_1}} & \cdots & \vec{x}_{EV_{n_1}}\\
            \vec{x}_{EV_{1_2}} & \vec{x}_{EV_{2_2}} & \cdots & \vec{x}_{EV_{n_2}}\\
            \vdots & \vdots & \ddots & \vdots\\
            \vec{x}_{EV_{1_n}} & \vec{x}_{EV_{2_n}} & \cdots & \vec{x}_{EV_{n_n}}\\
        \end{pmatrix}^{-1}
        $}&\notag\\
\end{flalign}