\subsubsection{Inverse berechnen mittels Adjunkter Matrix}
\begin{minipage}{0.5\linewidth}
    \textbf{Benötigt}\\
    \begin{itemize}
        \item Det(A) $\ne 0$
        \item 2x2 im Kopf | 3x3 mit Taschenrechner | 3x3 - nxn mit Adjunkte Matrix
    \end{itemize}
\end{minipage}
\hfill
\begin{minipage}{0.5\linewidth}
    \textbf{Wichtiges}\\
    \begin{itemize}
        \item Det(A) = 0 $\nRightarrow A^{-1}$\\ Det(A) $\ne 0 \Rightarrow A^{-1}$
        \item $(a \cdot A)^{-1} = \frac1a \cdot A^{-1}$
        \item $(A \cdot B)^{-1} = B^{-1} \cdot A^{-1}$
        \item $(A^T)^{-1} = (A^{-1})^T$
        \item $(A^{-1})^{-1} = A$
    \end{itemize}
\end{minipage}

\textbf{2x2 Matrix}\\
\begin{flalign}
    &A = \left[\begin{matrix}
        a & b\\
        c & d
    \end{matrix}\right] \qquad A^{-1} = \frac{1}{ad - bc} \cdot \left[\begin{matrix}
        d & -b\\
        -c & a
    \end{matrix}\right]&
\end{flalign}

\textbf{Anleitung 3x3, 4x4 - nxn}\\
\begin{flalign}
    &A^{-1} = \frac{1}{\det(A)} \cdot \scalebox{0.55}{$\displaystyle \left[
        \begin{matrix}
            {\color{red}\boxed{+}}
            \det(
            \begin{matrix}
                f & g & h\\
                j & k & l\\
                n & o & p
            \end{matrix}) &
            {\color{blue}\boxed{-}}
            \det(
            \begin{matrix}
                e & g & k\\
                i & k & l\\
                m & o & p
            \end{matrix}) &
            {\color{red}\boxed{+}}
            \det(
            \begin{matrix}
                e & f & h\\
                i & j & l\\
                m & n & p  
            \end{matrix}) &
            {\color{blue}\boxed{-}}
            \det(
            \begin{matrix}
                e & f & g\\
                i & j & k\\
                m & n & o
            \end{matrix})
            \\
            {\color{blue}\boxed{-}}
            \det(
            \begin{matrix}
                b & c & d\\
                j & k & l\\
                n & o & p
            \end{matrix}) &
            {\color{red}\boxed{+}}
            \det(
            \begin{matrix}
                a & c & d\\
                i & k & l\\
                m & o & p
            \end{matrix}) &
            {\color{blue}\boxed{-}}
            \det(
            \begin{matrix}
                a & b & d\\
                i & j & l\\
                m & n & p  
            \end{matrix}) &
            {\color{red}\boxed{+}}
            \det(
            \begin{matrix}
                a & b & c\\
                i & j & k\\
                m & n & o
            \end{matrix})
            \\
            {\color{red}\boxed{+}}
            \det(
            \begin{matrix}
                b & c & d\\
                f & g & h\\
                n & o & p
            \end{matrix}) &
            {\color{blue}\boxed{-}}
            \det(
            \begin{matrix}
                a & c & d\\
                e & g & h\\
                m & o & p
            \end{matrix}) &
            {\color{red}\boxed{+}}
            \det(
            \begin{matrix}
                a & b & d\\
                e & f & h\\
                m & n & p  
            \end{matrix}) &
            {\color{blue}\boxed{-}}
            \det(
            \begin{matrix}
                a & b & c\\
                e & f & g\\
                m & n & o
            \end{matrix})
            \\
            {\color{blue}\boxed{-}}
            \det(
            \begin{matrix}
                b & c & d\\
                f & g & h\\
                j & k & l
            \end{matrix}) &
            {\color{red}\boxed{+}}
            \det(
            \begin{matrix}
                a & c & d\\
                e & g & h\\
                j & k & l
            \end{matrix}) &
            {\color{blue}\boxed{-}}
            \det(
            \begin{matrix}
                a & b & d\\
                e & f & h\\
                i & j & l  
            \end{matrix}) &
            {\color{red}\boxed{+}}
            \det(
            \begin{matrix}
                a & b & c\\
                e & f & g\\
                i & j & k
            \end{matrix})
    \end{matrix}\right]^{\mathbf{T}}
    $}&\label{eq:4x4_Inverse_berechnen}
\end{flalign}

\begin{enumerate}
    \item Determinante  von A berechnen
    \item Determinanten der inneren Matrizen berechnen und das Ergebnis mit jeweiligem Vorzeichen eintragen (3x3 Spaltenweise mit TS ausrechnen)
    \item Matrix transponieren
    \item $A \cdot A^{-1} = \mathbb{I}$ prüfen
\end{enumerate}