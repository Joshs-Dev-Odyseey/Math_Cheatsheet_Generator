\subsubsection{Linearität}
\begin{flalign}
    &f(x) \cdot a = f(x \cdot a) \qquad \forall a \in \mathbf{R}, \forall x \in \mathbf{R}^{n}&\\
    &f(x) + f(y) = f(x + y) \qquad V_{x,y} \in \mathbf{R}^{n}&\\
    &\textbf{Lineares Beispiel: } f(x) = 3x \text{ mit } a=3,\ x=2,\ y=5&\notag\\
    &1.\ \text{Homogenitätstest: } f(2 \cdot 3) \stackrel{?}{=} 3 \cdot f(2) &\notag\\
    &\ \ \quad f(6) = 6 \cdot 3 = 18 \quad \text{vs.} \quad 3 \cdot 3 \cdot 2 = 3 \cdot 6 = 18 \quad \Rightarrow 18 = 18\ (\checkmark)&\notag\\
    &2.\ \text{Additivitätstest: } f(2) + f(5) \stackrel{?}{=} f(2+5) &\notag\\
    &\ \ \quad (3 \cdot 2) + (3 \cdot 5) = 6 + 15 = 21 \quad \text{vs.} \quad 3 \cdot 7 = 21 \quad \Rightarrow 21 = 21\ (\checkmark)&\notag\\
    &\textbf{Gegenbeispiel: } f(x) = x^2 \text{ mit } a=2,\ x=3,\ y=4&\notag\\
    &1.\ \text{Homogenitätstest: } f(3 \cdot 2) \stackrel{?}{=} 2 \cdot f(3) &\notag\\ 
    &\ \ \quad f(6) = 6^2 = 36 \quad vs. \quad 2 \cdot 3^2 = 18 \quad \Rightarrow 36 \ne 18\ (\times)&\notag\\
    &2.\ \text{Additivitätstest: } f(3) + f(4) \stackrel{?}{=} f(3+4) &\notag\\
    &\ \ \quad 3^2 + 4^2 = 9 + 16 = 25 \quad vs. \quad 7^2 = 49 \quad \Rightarrow 25 \ne 49\ (\times)&\notag
\end{flalign}

