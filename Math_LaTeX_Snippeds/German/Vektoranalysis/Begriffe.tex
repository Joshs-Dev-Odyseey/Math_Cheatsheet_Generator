\subsubsection{Begriffe}
\begin{itemize}
    \item Skalarfeld: Definiert was an jedem Punkt im Raum gemessen wird
    \item Levelmengen: Zeigen wo dieses Feld einen bestimmten Wert hat
    \item Gradientenfeld: Beschreibt, wie und wohin sich der Wert des Skalarfelds ändert
\end{itemize}

\subsubsection{Vektorfelder}
\begin{flalign}
    &\text{Einheitsvektorfeld: }\vec{v}(p) = \hat{v}(p)&\\
    &\text{Homogenes Vektorfeld: } \vec{v}(p) = \vec{w}&
\end{flalign}

\begin{flalign}
    \vec{v} = \underbrace{\nabla \phi}_{\text{Gradientenfeld (quellenfrei)}} + \underbrace{\nabla \times \vec{A}}_{\text{Rotationsfeld (wirbelfrei)}}
\end{flalign}
\begin{itemize}
    \item $\phi$: Skalarpotential (quellenfreier Anteil)
    \item $\vec{A}$: Vektorpotential (wirbelfreier Anteil)
    \item Zerlegung nach dem Helmholtz-Theorem
\end{itemize}