\subsubsection{Kurven Integral}
\begin{minipage}{0.5\linewidth}
    \begin{flalign}
        &\textbf{Skalares Kurvenintegral} &\notag\\
        &\int_{\gamma}{f} \,d\vec{s} = \int_{a}^{b}{f(\vec{\gamma}(t)) \cdot \left| \dot{\vec{\gamma}}(t) \right|} \,dt \label{eq:Skalares_Kurvenintegral}&\\
        &\textbf{Vektorielles Kurvenintegral}&\notag\\
        &\int_{\gamma}{\langle \vec{v}(t), d\vec{s} \rangle} = \int_{a}^{b}{ \langle \vec{v}(\vec{\gamma}(t)), \dot{\vec{\gamma}}(t) \rangle} \,dt \label{eq:Vektorielles_Kurvenintegral}&
    \end{flalign}
\end{minipage}
\hfill
\begin{minipage}{0.5\linewidth}
    \textbf{Benötigt}\\
    \begin{itemize}
        \item Funktion (Skalar) oder Vektorfeld (Vektoriell)
        \item Kurve
        \item Grenzen der Kurve
    \end{itemize}
    \textbf{Anleitung}
    \begin{enumerate}
        \item Kurve parameterisieren $\vec{v}$ oder $\vec{w}$
        \item Tangentialvektor $\vec{\gamma}$
        \item In 
        \item Sei $\langle \vec{w}, \hat{e} \rangle =: C \equiv$ konst dann gilt: $I = C \cdot \Delta s$
    \end{enumerate}
\end{minipage}


\subsubsection{Bogenlänge}
\begin{minipage}{0.4\linewidth}
    \begin{flalign}
        &\Delta s = \int_{a}^{b}{\left|\dot{\vec{\gamma}}(t)\right|} \,dt \label{eq:Kurven_Bogenlänge}&
    \end{flalign}
\end{minipage}
\hfill
\begin{minipage}{0.6\linewidth}
    \begin{enumerate}
        \item Gradient der Kurve ableiten
        \item Betrag des Gradienten berechnen (Satz des Pythagoras)
        \item In \ref{eq:Kurven_Bogenlänge} einsetzen und integrieren
    \end{enumerate}
\end{minipage}
