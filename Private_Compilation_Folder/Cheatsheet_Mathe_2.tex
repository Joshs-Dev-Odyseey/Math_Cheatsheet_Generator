\documentclass[a4paper, landscape, 6pt]{article}

% --------------------------------------------------------------------------------------------------------------------------------------------------------------------------
\input{./package_and_config.tex}

\def\subject{Mathematic}
\def\semester{1}
\def\author{Author}
\def\cols{3}

% Useful packages
\usepackage{amsmath}
\usepackage{graphicx}
\usepackage[colorlinks=true, allcolors=blue]{hyperref}
\usepackage{makecell}
\usepackage{adjustbox}
\usepackage{pgfplots}

\title{Title}

\cheatsheet{
    \section{\textbf{\textcolor{red}{This is an example Cheatsheet you could create with this Git Repo.}}}
    \section{Trigonometrie}
    \input{../Math_LaTeX_Snippeds/German/Tigonometrie/Unitcircle.tex}
    \input{../Math_LaTeX_Snippeds/German/Tigonometrie/unitcircle_2.tex}
    \input{../Math_LaTeX_Snippeds/German/Tigonometrie/Trigrometric_functions.tex}
    \input{../Math_LaTeX_Snippeds/German/Tigonometrie/Additionstheoreme.tex}
    \input{../Math_LaTeX_Snippeds/German/Tigonometrie/Parität.tex}
    \section{Allgemein} 
    \input{../Math_LaTeX_Snippeds/German/Allgemein/Geometrische_Formen.tex}
    \input{../Math_LaTeX_Snippeds/German/Allgemein/Taschenrechner.tex}
    \input{../Math_LaTeX_Snippeds/German/Allgemein/Funktionen.tex}
    \input{../Math_LaTeX_Snippeds/German/Allgemein/Ableiten.tex}
    \input{../Math_LaTeX_Snippeds/German/Allgemein/Linearitaet.tex}
    \input{../Math_LaTeX_Snippeds/German/Allgemein/Bijektiv_Surjektiv_Injektiv.tex}
    \input{../Math_LaTeX_Snippeds/German/Allgemein/Polynome.tex}
    \input{../Math_LaTeX_Snippeds/German/Allgemein/Hornerschema.tex}
    \input{../Math_LaTeX_Snippeds/German/Allgemein/Nullstellen_erraten.tex}
    \section{Analysis}
    \subsection{Integrale}
    \input{../Math_LaTeX_Snippeds/German/Integrale/Integration.tex}
    \input{../Math_LaTeX_Snippeds/German/Integrale/Uneigentliche_Integrale.tex}
    \input{../Math_LaTeX_Snippeds/German/Integrale/Substitution.tex}
    \input{../Math_LaTeX_Snippeds/German/Integrale/Standardintegrale.tex}
    \input{../Math_LaTeX_Snippeds/German/Integrale/Partielle_Integration.tex}
    \input{../Math_LaTeX_Snippeds/German/Integrale/Mehrfachintegrale.tex}
    \input{../Math_LaTeX_Snippeds/German/Integrale/Levelmengen.tex}
    \input{../Math_LaTeX_Snippeds/German/Integrale/Trapezformel.tex}
    \input{../Math_LaTeX_Snippeds/German/Integrale/Flaeche_berechnen.tex}
    \input{../Math_LaTeX_Snippeds/German/Integrale/Volumen_mit_rotation_einer_Funktion.tex}
    \input{../Math_LaTeX_Snippeds/German/Integrale/Schwerpunkt_berechnen.tex}
    \section{Komplexe Zahlen}
    \input{../Math_LaTeX_Snippeds/German/Komplexe_Zahlen/Koordinaten.tex}
    \input{../Math_LaTeX_Snippeds/German/Komplexe_Zahlen/Imaginaere_Zahlen.tex}
    \input{../Math_LaTeX_Snippeds/German/Komplexe_Zahlen/Arg_function.tex}
    \subsubsection{Koordinaten Wechsel}
\begin{flalign}
    &\textbf{Polarkoordinaten} & \notag \\[5pt]
    &\textbf{Polar zu Kartesisch} & \notag \\
    &x = r \cdot \cos\phi,\quad y = r \cdot \sin\phi & \\[5pt]
    &\textbf{Kartesisch zu Polar} & \notag \\
    &r = \sqrt{x^2 + y^2},\quad \tan\phi = \frac{y}{x} & \\[5pt]
    &\textbf{Basisvektoren} & \notag \\
    &\begin{bmatrix}
        \hat{e}_r \\
        \hat{e}_\phi
    \end{bmatrix} = 
    \begin{bmatrix}
        \cos\phi & \sin\phi \\
        -\sin\phi & \cos\phi
    \end{bmatrix}
    \begin{bmatrix}
        \hat{e}_x \\
        \hat{e}_y
    \end{bmatrix} & \\[5pt]
    &\textbf{Integration} & \notag \\
    &\mathrm{d}x\,\mathrm{d}y \mapsto r\,\mathrm{d}r\,\mathrm{d}\phi & \\[10pt]
\end{flalign}
\begin{flalign}
    &\textbf{Zylinderkoordinaten} & \notag \\[5pt]
    &\textbf{Zylinder zu Kartesisch} & \notag \\
    &x = \rho\cos\phi,\quad y = \rho\sin\phi,\quad z = z & \\[5pt]
    &\textbf{Kartesisch zu Zylinder} & \notag \\
    &\rho = \sqrt{x^2 + y^2},\quad \sin\phi = \frac{y}{\rho},\quad z = z & \\[5pt]
    &\textbf{Basisvektoren} & \notag \\
    &\begin{bmatrix}
        \hat{e}_\rho \\
        \hat{e}_\phi \\
        \hat{e}_z
    \end{bmatrix} =
    \begin{bmatrix}
        \cos\phi & \sin\phi & 0 \\
        -\sin\phi & \cos\phi & 0 \\
        0 & 0 & 1
    \end{bmatrix}
    \begin{bmatrix}
        \hat{e}_x \\
        \hat{e}_y \\
        \hat{e}_z
    \end{bmatrix} & \\[5pt]
    &\textbf{Integration} & \notag \\
    &\mathrm{d}A\,\mathrm{d}z \mapsto \rho\,\mathrm{d}\phi\,\mathrm{d}\rho\,\mathrm{d}z &
\end{flalign}  
\begin{flalign}
    &\textbf{Kugelkoordinaten} & \notag \\[5pt]
    &\textbf{Kugel zu Kartesisch} & \notag \\
    &x = r\sin\theta\cos\phi,\ y = r\sin\theta\sin\phi,\ z = r\cos\theta & \\[5pt]
    &\textbf{Kartesisch zu Kugel} & \notag \\
    &r = \sqrt{x^2 + y^2 + z^2},\ \tan\phi = \frac{y}{x},\ \cos\theta = \frac{z}{r} & \\[5pt]
    &\textbf{Basisvektoren} & \notag \\
    &\begin{bmatrix}
        \hat{e}_r \\
        \hat{e}_\theta \\
        \hat{e}_\phi
    \end{bmatrix} = 
    \begin{bmatrix}
        \sin\theta\cos\phi & \sin\theta\sin\phi & \cos\theta \\
        \cos\theta\cos\phi & \cos\theta\sin\phi & -\sin\theta \\
        -\sin\phi & \cos\phi & 0
    \end{bmatrix}
    \begin{bmatrix}
        \hat{e}_x \\
        \hat{e}_y \\
        \hat{e}_z
    \end{bmatrix} & \\[5pt] 
    &\textbf{Integration} & \notag \\
    &\mathrm{d}V \mapsto r^2\sin\theta\,\mathrm{d}r\,\mathrm{d}\theta\,\mathrm{d}\phi &
\end{flalign}
    \section{Lineare Algebra}
    \subsection{Vektoranalysis}
    \input{../Math_LaTeX_Snippeds/German/Vektoranalysis/Begriffe.tex}
    \input{../Math_LaTeX_Snippeds/German/Vektoren/Normalenvektor.tex}
    \input{../Math_LaTeX_Snippeds/German/Vektoren/Kreuzprdukt.tex}
    \input{../Math_LaTeX_Snippeds/German/Vektoranalysis/Parametriesierte_Kurve.tex}
    \input{../Math_LaTeX_Snippeds/German/Vektoranalysis/Kurvenintegral.tex}
    \input{../Math_LaTeX_Snippeds/German/Vektoranalysis/Standard_Kurven.tex}
    \input{../Math_LaTeX_Snippeds/German/Vektoranalysis/Gradient.tex}
    \input{../Math_LaTeX_Snippeds/German/Vektoranalysis/Hessematrix.tex}
    \input{../Math_LaTeX_Snippeds/German/Vektoranalysis/Richtungsableitung.tex}
    \input{../Math_LaTeX_Snippeds/German/Vektoranalysis/Divergenz.tex}
    \subsubsection{Rotation}
\textbf{Wichtiges}\\
\begin{itemize}
    \item rot$(\vec{x}) = 0 \rightarrow$ konservativ
\end{itemize}
\begin{minipage}{0.4\linewidth}
    \begin{flalign}
        &\textbf{Rotation}&\notag\\
        &\operatorname{rot}{\vec{v}} = \vec{v}^{1}_{,1} - \vec{v}^{1}_{,2}&
    \end{flalign}
\end{minipage}
\hfill
\begin{minipage}{0.4\linewidth}
    \begin{flalign}
        &\textbf{Rotation in 3D}&\notag\\
        &\operatorname{rot}{v} = \left[\begin{matrix} v^{3}_{,2} - v^{2}_{,3}\\
            v^{1}_{,3} - v^{3}_{,1}\\
            v^{2}_{,1} - v^{1}_{,2}\\
        \end{matrix}\right]&
    \end{flalign}
\end{minipage}
    \input{../Math_LaTeX_Snippeds/German/Vektoranalysis/Tangentialebene.tex}
    \input{../Math_LaTeX_Snippeds/German/Vektoranalysis/Totales_Differential.tex}
    \subsection{Matrizen}
    \input{../Math_LaTeX_Snippeds/German/Matrizen/Begriffe.tex}
    \subsubsection{Extremwertstellen/Kritische Stellen}

\begin{flalign}
    &\boxed{1}: \text{Nebenbedingung nach 0 umstellen}&\notag\\
    &\det(\begin{matrix}
        f_x & g_x\\
        f_y & f_y\\
    \end{matrix}) = \begin{cases}
        \boxed{2}: L(\vec{x}, \lambda) = f(\vec{x}) + \lambda \cdot g(\vec{x}) + \cdots\\
        \boxed{3}: \nabla L \stackrel{!}{=} 0\\
        \boxed{4}: \lambda \text{eliminieren mit Additionsverfahren}
    \end{cases}&\\
    &\boxed{5}: \text{Gleichungssystem lösen}\notag&
\end{flalign}


    \subsubsection{Definitheit}
\textbf{Fälle}\\
\begin{enumerate}
    \item Positiv definit $\Delta_k > 0$ (++++) $\rightarrow$ Minimum
    \item Negativ definit $(-1)^k \cdot \Delta_k > 0$ (-+-+) $\rightarrow$ Maximum
    \item Indefinit (+00+0+ | 0+-+0 | ---- | +-+- ) $\rightarrow$ Sattelpunkt
\end{enumerate}

\textbf{Anleitung}\\
\begin{enumerate}
    \item Kritischen Punkt in Hessematrix einsetzen
    \item Definitheit mit Minorenverfahren bestimmen
\end{enumerate}
    \input{../Math_LaTeX_Snippeds/German/Matrizen/Standard_Matrizen.tex}
    \input{../Math_LaTeX_Snippeds/German/Matrizen/Spaltenkonstruktions_Verfahren.tex}
    \input{../Math_LaTeX_Snippeds/German/Matrizen/Determinante.tex}
    \input{../Math_LaTeX_Snippeds/German/Matrizen/Inverse_berechnen_Adjunkter_Matrix.tex}
    \input{../Math_LaTeX_Snippeds/German/Matrizen/Eigenwerte.tex}
    \input{../Math_LaTeX_Snippeds/German/Matrizen/Diagonalisierung.tex}
}
\end{document}